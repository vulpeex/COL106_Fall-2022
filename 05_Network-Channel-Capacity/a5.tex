\documentclass{article}
\usepackage{amsmath}
\usepackage{amssymb}
\usepackage{amsthm}
\usepackage{fancyhdr}  % Package for custom headers and footers
\usepackage[a4paper, margin=0.9in, ]{geometry}  % Adjust margins here
\pagestyle{fancy}  % Activate fancy page style
\fancyhf{}  % Clear default header and footer
% Customize header
\fancyhead[L]{COL106\vspace{5pt}\\Release Date: 2022/10/??}  % Left side header
\fancyhead[R]{Assignment 5\vspace{5pt}\\Deadline: 2022/10/?? 23:00}  % Right side header
% Customize footer
\fancyfoot[C]{\thepage}  % Center footer (page number)
\begin{document}
\section*{Motivation}
Consider a simple abstraction of a computer network consisting of $n$ routers communicating information to one another. The communication is established by connecting some pairs of routers using links in such a way that for any pair of routers $x$ and $y$, it is possible to send data from $x$ to $y$ through a sequence of links. To send a large file over a network, it is customary to divide it into smaller packets, and send those packets one-by-one. The choice of the packet size is determined by the underlying network: if a packet is larger than the capacity of a link, the link is unable to transfer the packet and simply drops it. On the other hand, having too small packets is undesirable due to a per-packet overhead involved in the transfer. Therefore, given a source and a destination in the network, it makes sense to determine the size of the largest packet that can be transferred through the network from the source to the destination.

\section*{Problem Definition}
Let $n$ denote the number of routers in our network, and let’s say that the routers are labeled $0, \dots, n - 1$. Each link is specified by a tuple $(u, v, c)$, where $u$ and $v$ are the endpoints of the link and $c$ is its capacity. Such a link is able to transfer a packet of size at most $c$ from $u$ to $v$ and from $v$ to $u$ (every link is bidirectional and has the same capacity in either direction). Given the identities $s$ and $t$ of the source and the target router respectively, we wish to determine the largest $C$ such that a packet of size $C$ can be transferred over the network from $s$ to $t$. Your job is to write a program to find that $C$ and a sequence of routers $v_0, v_1, \dots, v_{r-1}$, where $s = v_0$ and $v_{r-1} = t$, such that for each $i$, there exists a link of capacity at least $C$ between $v_{i-1}$ and $v_i$.

Let $m$ denote the number of links in the network. You may assume that the network is connected, that is, it is possible to transfer a small enough packet from any node to any other node. Note that a pair of routers can have zero, one, or more than one links of different capacities between them. For full credit, your program must run in time $O(m \log m)$. (In fact, it is possible to do this in time $O(m)$. Figure out how in the vacation following the major exam.)

\section*{Submission Specifications}
Submit a single file named \texttt{a5.py}. Your file must contain the implementation of a function \texttt{findMaxCapacity}. The function must take the following as parameters in the given order.
\begin{enumerate}
    \item $n$: the number of routers in the network,
    \item \texttt{links}: a list of 3-tuples of integers. A tuple $(u, v, c)$ in this list represents a bidirectional link of capacity $c$ between $u$ and $v$, where $u, v \in \{0, \dots, n-1\}$,
    \item $s$: the source router ($s \in \{0, \dots, n-1\}$),
    \item $t$: the target router ($t \in \{0, \dots, n-1\}$).
\end{enumerate}

The function must return a pair $(C, \texttt{route})$, where
\begin{enumerate}
    \item $C$ is the largest number such that a packet of size $C$ can be transferred over the network from $s$ to $t$,
    \item \texttt{route} is a list of numbers from the set $\{0, \dots, n-1\}$ such that \texttt{route}[0] is $s$, \texttt{route}[\texttt{len}(\texttt{route})-1] is $t$, and for each $i$, there exists a link of capacity at least $C$ between \texttt{route}[$i-1$] and \texttt{route}[$i$].
\end{enumerate}

\section*{Example Test Cases}
\begin{verbatim}
>>> findMaxCapacity(3,[(0,1,1),(1,2,1)],0,1)
(1,[0,1])

>>> findMaxCapacity(4,[(0,1,30),(0,3,10),(1,2,40),(2,3,50),(0,1,60),(1,3,50)],0,3)
(50,[0,1,3])

>>> findMaxCapacity(4,[(0,1,30),(1,2,40),(2,3,50),(0,3,10)],0,3)
(30,[0,1,2,3])

>>> findMaxCapacity(5,[(0,1,3),(1,2,5),(2,3,2),(3,4,3),(4,0,8),(0,3,7),(1,3,4)],0,2)
(4,[0,3,1,2])

>>> findMaxCapacity(7,[(0,1,2),(0,2,5),(1,3,4),(2,3,4),(3,4,6),(3,5,4),(2,6,1),(6,5,2)],0,5)
(4,[0,2,3,5])
\end{verbatim}


\thispagestyle{plain}
\end{document}