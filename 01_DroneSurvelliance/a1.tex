\documentclass{article}
\usepackage{amsmath}
\usepackage{amssymb}
\usepackage{amsthm}
\usepackage{fancyhdr}  % Package for custom headers and footers
\usepackage[a4paper, margin=0.9in, ]{geometry}  % Adjust margins here
\pagestyle{fancy}  % Activate fancy page style
\fancyhf{}  % Clear default header and footer
% Customize header
\fancyhead[L]{COL106\vspace{5pt}\\Release Date: 2022/08/17}  % Left side header
\fancyhead[R]{Assignment 1\vspace{5pt}\\Deadline: 2022/08/31 23:00}  % Right side header
% Customize footer
\fancyfoot[C]{\thepage}  % Center footer (page number)
\begin{document}
\section{Problem Statement}
IIT Delhi has recently deployed a drone for aerial surveillance on account of Independence Day. The drone begins at position $(0,0,0)$ and can move infinitely in any direction depending on how it is programmed. A drone program is a string consisting of a sequence of instructions, where each instruction is one of the following:
\begin{itemize}
    \item $+X$: Move one unit in the direction of the positive X-axis.
    \item $-X$: Move one unit in the direction of the negative X-axis.
    \item $+Y$: Move one unit in the direction of the positive Y-axis.
    \item $-Y$: Move one unit in the direction of the negative Y-axis.
    \item $+Z$: Move one unit in the direction of the positive Z-axis.
    \item $-Z$: Move one unit in the direction of the negative Z-axis.
    \item $m(P)$, where $m > 0$ is an integer and $P$ is a drone program: Execute program $P$ $m$ times.
\end{itemize}

For example:
\begin{itemize}
    \item $2(+X+Y-Z)$ is equivalent to $+X+Y-Z+X+Y-Z$, moving the drone to $(2, 2, -2)$ after traveling a distance of 6.
    \item $5(+X)10(-X)$ is equivalent to $+X+X+X+X+X-X-X-X-X-X-X-X-X-X-X$, moving the drone to $(-5, 0, 0)$ after traveling a distance of 15.
    \item $3(-Y2(+Z))$ is equivalent to $-Y+Z+Z-Y+Z+Z-Y+Z+Z$, moving the drone to $(0, -3, 6)$ after traveling a distance of 9.
    \item $+X+X+X+X4(+Y)2(+Z-Z)$ is equivalent to $+X+X+X+X+Y+Y+Y+Y+Z-Z+Z-Z$, moving the drone to $(4, 4, 0)$ after traveling a distance of 12.
\end{itemize}

Your task is to write a Python program that takes a drone program $P$ as input and outputs the following:
\begin{enumerate}
    \item The final position of the drone after it has executed its program $P$.
    \item The total distance travelled by the drone in this process.
\end{enumerate}

To solve this problem, it will be helpful to use stacks, so implement the \texttt{Stack} data structure and its member functions from scratch. For full credit, your program must run in time $O(n)$ on drone programs of length $n$. We will assume that your program runs in linear time provided it terminates within a specific amount of time set by our auto-grader. To minimize the possibility of a false auto-grader timeout, you are advised to remove all unnecessary print statements that you might have written to debug your program.

\section{Submission Specifications}
Submit a single file named \texttt{a1.py}. Your submitted file must contain a function \texttt{findPositionandDistance(P)} that takes a string $P$ as input and returns a list $[x, y, z, d]$ containing four numbers, where $(x, y, z)$ is the final position of the drone after it executes program $P$, and $d$ is the distance traveled by the drone in the process.

\section{Example Test Cases}
\begin{verbatim}
>>> findPositionandDistance('+X+Y+X-Y-Z+X+X-Z-Z-Z-Z-Y+Y-X')
[3, 0, -5, 14]

>>> findPositionandDistance('+X2(+Y-X-Z)8(+Y)9(-Z-Z)')
[-1, 10, -20, 33]

>>> findPositionandDistance('')
[0, 0, 0, 0]

>>> findPositionandDistance('5(2(3(+X+X)))')
[60, 0, 0, 60]

>>> findPositionandDistance('+Z6(+X+Y+X-Y)9(-X+Z-X-Z8(+X+Y-Z)9(+Y-Z-X-Y4(-X+Y-X-Z+X)))')
[-339, 396, -476, 2221]

>>> findPositionandDistance('1(+X)5(+Y)41(+Z)1805(-X)3263441(-Y)10650056950805(-Z)')
[-1804, -3263436, -10650056950764, 10650060216098]
\end{verbatim}
\thispagestyle{plain}
\end{document}